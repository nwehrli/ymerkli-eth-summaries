\documentclass[11pt,landscape,a4paper,fleqn]{article}
\usepackage[utf8]{inputenc}
\usepackage[ngerman]{babel}
\usepackage{tikz}
\usetikzlibrary{shapes,positioning,arrows,fit,calc,graphs,graphs.standard}
\usepackage[nosf]{kpfonts}
\usepackage[t1]{sourcesanspro}
%\usepackage[lf]{MyriadPro}
%\usepackage[lf,minionint]{MinionPro}
\usepackage{multicol}
\usepackage{wrapfig}
\usepackage[top=0mm,bottom=3mm,left=1mm,right=1mm]{geometry}
\usepackage[framemethod=tikz]{mdframed}
\usepackage{microtype}
\usepackage{mathtools}
\usepackage{ccicons}
\usepackage{hyperref}
\usepackage[inline]{enumitem}
% \PassOptionsToPackage{dvipsnames}{xcolor}
\usepackage{xcolor}
% \usepackage[dvipsnames]{xcolor}

\let\bar\overline
\def\D{\mathcal{D}}

\definecolor{myblue}{cmyk}{1,.72,0,.38}
\definecolor{myorange}{cmyk}{0,0.5,1,0}

\pgfdeclarelayer{background}
\pgfsetlayers{background,main}

\everymath\expandafter{\the\everymath \color{myblue}}
\everydisplay\expandafter{\the\everydisplay \color{myblue}}

\renewcommand{\baselinestretch}{.8}
\pagestyle{empty}

\global\mdfdefinestyle{header}{%
linecolor=gray,linewidth=1pt,%
leftmargin=0mm,rightmargin=0mm,skipbelow=0mm,skipabove=0mm,
}

\newcommand{\header}{
\begin{mdframed}[style=header]
\footnotesize
\sffamily
Cheat sheet\\
Nicolas Wehrli,~page~\thepage~of~2
\end{mdframed}
}

\newcommand{\middot}{~\textperiodcentered~}
\newlist{rowlist}{enumerate*}{1}
\setlist[rowlist]{label={\textbf{\roman*}\text{: }}, afterlabel={}, itemjoin=\middot}

\newcommand{\E}[0]{\mathbb{E}}
\newcommand{\N}[0]{\mathbb{N}}
\newcommand{\R}[0]{\mathbb{R}}

\newcommand{\sgn}[0]{\text{sgn}}

\newcommand{\argmin}[1]{\underset{#1}{\text{argmin}}}
\newcommand{\argmax}[1]{\underset{#1}{\text{argmax}}}


\makeatletter
\renewcommand{\section}{\@startsection{section}{1}{0mm}%
                                {0pt}%
                                {0.5pt}%x
                                {\color{myorange}\sffamily\small\bfseries}}
\renewcommand{\subsection}{\@startsection{subsection}{1}{0mm}%
                                {0pt}%
                                {0.1pt}%x
                                {\sffamily\bfseries}}


\makeatother
\setlength{\parindent}{0pt}

\newcommand{\red}[1]{\textcolor{red}{#1}}
\newcommand\iid{\stackrel{\mathclap{\normalfont\mbox{\tiny{iid}}}}{=}}

\begin{document}
	
\section*{Disclaimer}

\small
\begin{multicols*}{4}
	
This document is an exam summary that follows the slides of the \textit{Introduction to Machine Learning} lecture  at ETH Zurich. 
The contribution to this is a short summary that includes the most important concepts, formulas and algorithms. 
This summary was created during the spring semester 2018 by Yannik Merkli and adapted in 2024 by Nicolas Wehrli. 
Due to updates to the syllabus content, some material may no longer be relevant for future versions of the 
lecture. This work is published as CC BY-NC-SA.

\begin{center}
	\ccbyncsa
\end{center}

I do not guarantee correctness or completeness, nor is this document endorsed by the lecturers. 
Feel free to point out any erratas. 
For the full \LaTeX \ source code, consider \texttt{\href{https://github.com/nwehrli/ymerkli-eth-summaries}{https://github.com/nwehrli/ymerkli-eth-summaries}}.

\newpage
	
%\input{Probabilities}
\section*{Basics}
%-------------------------------------------------------------------------------------------
%PROBABILITIES
%-------------------------------------------------------------------------------------------
% $P(X,Y)=P(X|Y)P(Y)=P(Y|X)P(X)\\
% P(X,Y|Z)=P(X|Y,Z)P(Y|Z)\\
% P(X,Y|Z)=P(Y|X,Z)P(X|Z)$\\
% $P(X,Y,Z)=P(X|Y,Z)P(Y|Z)P(Z)$\\
% $\text{X,Y iid:}P(X,Y|Z)=P(X|Z)P(Y|Z)$\\
% $P(X=x)=\sum_{y'\in Y}P(X=x,Y=y')$\\
% $\text{iid:}P(X_1,...,X_n|Y)=\prod_{i=1}^{n}P(X_i|Y)$\\
% $\mathbb{E}_x[X] = \begin{cases}
% \int x \cdot p(x) \partial x  &|\mathbb{E}_x[f(x)] =\\
% \sum_x x \cdot p(x) &|\int f(x) \cdot p(x) \partial x
% \end{cases}$\\
% %$\mathbb{E}_x[f(x)] = \int f(x) \cdot p(x) \partial x $\\
% $\mathbb{E}[X+Y]=\mathbb{E}[X]+\mathbb{E}[Y]$\\
% $\sigma_X^2=Var[X] = \mathbb{E}[(X-\mu_X)^2] = \mathbb{E}[X^2] - \mathbb{E}[X]^2$\\
% $p(Z|X,\theta) = \frac{p(X,Z|\theta)}{p(X|\theta)}$
%\subsection*{Linearity of expectation}
%$X, Y$ rand. var., $a, b \in \mathbb{R}$:\\
%$\mathbb{E}_{x,y}[aX + bY] = a\mathbb{E}_x[X] + b \mathbb{E}_y[Y]$

%-------------------------------------------------------------------------------------------
%Calculus and stuff
%-------------------------------------------------------------------------------------------


% $ln(x) \leq x - 1, x>0$; $||x||_2 = \sqrt{x^T x}$; $\nabla_x ||x||_2^2 = 2 x$%; $||x||_p = (\sum_{i=1}^n|x_i|^p)^{\frac{1}{p}}$, $1 \leq p < \infty$

% $f(x) = x^T A x$; $\nabla_x f(x) = (A + A^T) x$

% $D_{KL} = \mathbb{E}_p[log(\frac{p(x)}{q(x)})]$; $D_{KL} (P||Q) = \sum_{x \in X}P(x) \cdot log \frac{P(x)}{Q(x)} =  \int_{-\infty}^{+\infty} p(x) log \frac{p(x)}{q(x)} \, dx $ always nonneg

\textbf{Orth:} A: $det(A)\in\{+1,-1\},AA^T=A^TA=I$\\
\(trace(ABC) = trace(BCA) = trace(CAB)\)\\ 
\(trace(A)=\sum \lambda_i(A);
% $, A\in\mathbb{R}^{n\times n}, (A^{-1})^T=(A^T)^{-1}$\\
% $rank(A)=n, det(A)\neq0$\\
\begin{bmatrix}
a&b \\ 
c&d
\end{bmatrix}^{-1}=\frac{1}{ad-bc}
\begin{bmatrix}
d&-b \\ 
-c&a
\end{bmatrix}
\)

\(A=\sum_{k=1}^{rk(A)}\sigma_{k,k}u_k (v_k)^T, A^\dag = U S' V^T; \sigma'_{k, k} = \frac{1}{\sigma_{k,k}}\)

\textbf{Deriv:}
$\frac{\partial}{\partial x}b^Tx=\frac{\partial}{\partial x}x^Tb=b^T,\!
\frac{\partial}{\partial x}||x||_2^2=2x^T,\! \frac{\partial}{\partial x} ||x -a||_2 = \frac{(x-a)^T}{||x-a||_2},\!
\frac{\partial}{\partial x}(x^TAx)=x^T(A^T+A),$
$\frac{\partial}{\partial x}(b^TAx)=A^Tb, \nabla_X(c^TXb)=cb^T,
\nabla_X(c^TX^Tb)=bc^T; ||A||_{op} = sup_{||x||_2=1}||Ax||_2$

$\textbf{convex} \iff f(\lambda x+ (1-\lambda)y) \leq \lambda f(x) + (1-\lambda)f(y); 
f(y) \geq f(x) + \langle\nabla f(x), y-x\rangle; D^2f(x) \geq 0$\\
\(\alpha f + \beta g \textbf{ c.}; \max(f, g) \textbf{ c.} \text{ if } f, g \textbf{ c.}, \alpha, \beta \geq 0\)\\
\(f \circ g = f(g(x)) \textbf{ c.} \text{ if } f \textbf{ c.}, g \textbf{ a.} \lor f \textbf{c., non-dec.}, g \textbf{ c.}\)
% \textbf{Eigdec:}
% $A,Q \in \mathbb{R}^{n\times n}, A=Q\Lambda Q^{-1},\! \Lambda = diag(\lambda_i)$\\
% $Q=[v_1,..,v_n], \text{(col's are e-vec.)}$\\
% $\text{if all $\lambda_i\geq0:$} A^{-1}=Q\Lambda^{-1}Q^{-1},\Lambda^{-1}=diag(\frac{1}{\lambda_i})$\\
% $\text{if }A=A^T\text{(symm.) and }x^TAx\geq0 \forall x \neq 0 \rightarrow psd$\\

% $X\in \mathbb{R}^{n\times p}, U\in \mathbb{R}^{n\times n}, S\in \mathbb{R}^{n\times p},
% V\in \mathbb{R}^{p\times p}$\\
% $X^TX=VS^TU^TUSV^T=VS^TSV^T=V\Sigma V^T$\\
% $\Sigma = diag(\sigma_1^2,..,\sigma_n^2);\sigma_i^2=\lambda_i; \forall \lambda_i \geq 0$
\(p_{\mu, \Sigma}(x)=\frac{1}{\sqrt{(2\pi)^p det(\Sigma)}}\exp(-\frac{1}{2}(x-\mu)^T\Sigma^{-1}(x-\mu))\)\\
\(\mathbf{X} \sim \mathcal{N}(\mu, \Sigma) \implies A\mathbf{X} + b \sim \mathcal{N}(A\mu+b, A\Sigma A^T)\)

% \textbf{Gauss CDF:}\\
% $\Phi(u;v,w) = \int_{-\infty}^{u} \mathcal{N}(y;v,w)dy=\Phi(\frac{u-v}{\sqrt{w}};0,1)$;\\ %CDF: cumulative distribution function; PDF: standard normal probability density function, $\mu = 0$, $\sigma = 1$
%PDF: $\phi(x) = \frac{1}{\sqrt{2\pi}} e^{-(1/2)x^2}$; $\int \phi(x) \partial x = \Phi(x) + c$;\\
%$\int x \phi(x) = -\phi(x) + c$; $\int x^2 \phi(x) \partial x = \Phi(x) -x \phi(x) + c$
\textbf{Jensen ineq: }$g(E[X]) \leq E[g(X)]$, $g$ convex


\input{chapters/02_Regression.tex}
\section*{Classification}
$\hat{y}=sign(f(x))=sign(w^Tx), z = yf(x)$
Surrogate Losses for 0-1: \(\ell_{\text{exp}(z)} = e^{-z}, \ell_{\text{log}}(z) = \log(1+e^{-z})\)

\(\nabla_z \ell_{\text{exp}}\) explodes for \(z \to -\infty\), sens. to outliers.
\textbf{Logistic Reg.} \(L(w) = \frac{1}{n}\sum_{i = 1}^n \log(1+e^{-y_iw^Tx_i})\)
(linear boundary!)


% Perceptron\\
% $l_{P} (w;y_i,x_i) = max\{0, -y_i w^T x_i \}$\\
% $w^* = \operatorname{argmin_w} \sum_{i=1}^n l_p (w;y_i,x_i)$\\
% $\nabla_w l_p(w;y_i,x_i) = (-y_ix_i)1[y_iw^Tx_i<0]$

% \subsection*{Stochastic Gradient Descent (SGD)}
% 1. Start at an arbitrary $w_0 \in \mathbb{R}^d$\\
% 2. For $t = 1, 2,  ...$ do: \\
% 	Pick data point $(x',y') \in_{u.a.r.} D$\\
% 	$w_{t+1} = w_t - \eta_t \nabla_w l(w_t;x',y')$\\
% Perceptron Alg: SGD with Perceptron loss

%\subsection*{Perceptron Algorithm}
%Stoch. Gradient Descent with Perceptron loss\\
%\emph{Theorem:} If $D$ is linearly separable $\Rightarrow$ Perceptron will obtain a linear separator.

%\subsection*{Hinge loss}
%Loss for Support Vector Machine.\\
%$l_H(w;x,y) = max \{0,1-y w^T x\}$

\subsection*{MM and SVM}
\(w_{\text{MM}} = {\arg \max}_{||w||_2=1}\min_{1\leq i \leq n} y_i \langle w, x_i \rangle\)\\
\(w_{\text{SVM}} = \arg \min||w||_2 \text{ s.t. } y_i\langle w, x_i \rangle \geq 1, \forall i\)

If data \textbf{linearly sep.}, 1. \(w_{\text{SVM}} = w_{\text{MM}}||w_{\text{SVM}}||_2\)

2. GD on logistic reg. (\(\eta = 1\)): \(\frac{w^t}{||w^t||_2} \to w_{\text{MM}}\) 

If \textbf{not} and \(ker(X) = \emptyset\), GD on logistic reg. (\(\eta = \frac{4}{\lambda_{max}(X)}\))
\(w^t \to \hat{w}\), \(\hat{w}\) global min.

\textbf{Hinge loss}: $l_H(w;x,y) = max \{0,1-y w^T x\}$\\
\textbf{soft-mar.}
$w^* = \underset{w}{\operatorname{argmin}} ||w||_2^2 + \lambda\sum_{i=1}^nl_H(w;x_i,y_i)\\
g_i(w) = max \{0,1-y_i w^T x_i\} + \lambda ||w||_2^2\\
\nabla_w g_i(w) = \begin{cases}
    -y_i x_i + 2\lambda w &\text{ , if $y_i w^T x_i<1$}\\
		2\lambda w &\text{ , if $y_i w^T x_i \geq 1$}
\end{cases}$

%\subsection*{L1-SVM}
%$\underset{w}{\operatorname{min}} \lambda ||w||_1 + \sum_{i=1}^n max(0,1-y_i w^T x_i)$

%\subsection*{Matrix-Vector Gradient}
%multiply transposed matrix to the same side as its occurance w.r.t. derivate variable: $\beta \in \mathbb{R}^d$
%$\nabla_\beta ( ||y-X\beta||_2^2 + \lambda ||\beta||_2^2 ) = 2X^T (y-X\beta) + 2\lambda \beta$\\

\subsection*{Multi-Class Classification}
\(\hat{y}(x)=\operatorname{argmax_{k \in \{1,.,K\}}}f_k(x)\); \(f = (f_1, ..., f_K)\)\\
\textbf{OvR}: For each class \(k \in [K]\):

1. Relabel \(\tilde{y}_i = 1\) if \(y_i = k\), else \(\tilde{y}_i = -1\) as \(\D_k\)

2. Train \(f_k\) as binary classifier on \(\D_k\) 

\textbf{Cross-E. Loss}: \(\ell_{\text{ce}}(f(x), y) = -\log\left(\frac{e^{f_y(x)}}{\sum_{k=1}^Ke^{f_k(x)}}\right)\)

\textbf{OvO}: \(L(w) = \sum_{i = 1}^n\ell_{\text{ce}}(f_w(x_i), y_i)\); GD on \(L(w)\)

Multiplicative noise model: \(y = y^*(x)\varepsilon\)
\subsection*{Cost Sensitive Classification}
Replace loss by: $l_{CS}(w;x,y) = c_y l(w;x,y)$

\subsection*{Metrics (convention: positive = rare)}
Accuracy=$\frac{\text{\#correct predictions}}{\#all\, predictions}$=$\frac{TP+TN}{TP+TN+FP+FN}$, 
Precision=$\frac{\#correct'+'predictions}{\#all'+'predictions}$=$\frac{TP}{TP+FP}=$1-FDR\\
Rec.=TPR=$\frac{TP}{TP+FN}=\frac{TP}{n_+}$, FPR=$\frac{FP}{TN+FP}=\frac{FP}{n_-}$=T1\\
F1 score $=\frac{2TP}{2TP+FP+FN}=\frac{2}{\frac{1}{Precision}+\frac{1}{Recall}}$

\(\hat{y_\tau}(x) = \text{sign}(\hat{f}(x) - \tau)\) (varying threshold)
\section*{Kernels $k: \mathcal{X} \times \mathcal{X} \rightarrow \mathbb{R}$}

Reparam. \(w = \phi^T\alpha\); \(f(x) = \sum_{i=1}^n\alpha_i\langle\phi(x_i), \phi(x)\rangle\)

A kernel is \textbf{valid} if $K$ is sym.: $k(x,z) = k(z,x)$ and psd: $z^\top K z \geq 0$

\textbf{mono.}: $k(x,y) = (x^\top y)^m$,
\textbf{poly}: $k(x,y) = (1+x^\top y)^m$,
\textbf{RBF}: $k(x, z) = \exp ( -\frac{||x - z||_\alpha}{\tau} )$, $\alpha = 1 \Rightarrow $ Laplacian, $\alpha = 2 \Rightarrow $ Gaussian

\textbf{Mercers Theorem}: Valid kernels can be decomposed into a lin. comb. of inner products.

\textbf{Kernel composition}
$k = k_1 + k_2$, 
$k = k_1 \cdot k_2$,
$\forall c > 0. \; k = c \cdot k_1$,
$k = f(k_1)$, $f$ pwr. series w/ non-neg. coeffs.,
$k(\binom{x}{y}, \binom{x'}{y'})=k(x,x')k(y,y')$, $k(\binom{x}{y}, \binom{x'}{y'})=k(x,x') + k(y,y')$,
$k(x, x') = g(\left< x, x' \right>)$, $g$ all Taylor coefficients non-negative,
$\forall f. \; k(x,y) = f(x)k_1(x,y)f(y)$

\textbf{Kern. Ridge Reg.}
$\min_w \frac1n \|y - \Phi w\|^2 + \lambda ||w||_2^2 = \min_\alpha \frac1n ||y - K\alpha||_2^2 + \lambda \alpha^\top K \alpha$


\subsection*{k Nearest Neighbor classifier}
% $y=sign(\sum_{i=1}^n y_i [x_i \text{ among k nn of } x])$
\begin{rowlist}
	\item Pick $k$ and distance metric $d$
	\item For given $x$, find among $x_1,...,x_n \in D$ the $k$ closest to $x \to x_{i_1},..., x_{i_k}$
	\item Output the majority vote of labels
\end{rowlist}

% \input{chapters/05_Multi-class}
\input{chapters/06_NeuralNetworks}
\section*{Clustering}
\subsection*{k-mean}
$\hat{R}(\mu) = \sum_{i=1}^n \underset{j\in\{1,...k\}}{\operatorname{min}}||x_i-\mu_j||_2^2$, 
non-convex, NP-hard, kernelizable, local opt., spherical bias\\
\textbf{Algorithm (Lloyd's heuristic):}\\
Initialize cluster centers $\mu^{(0)} = [\mu_1^{(0)},...,\mu_k^{(0)}]$\\
While still changes in assignments:\\
$z_i^{(t)} = \underset{j\in\{1,...,k\}}{\operatorname{argmin}}||x_i - \mu_j^{(t-1)}||_2^2$; $\mu_j^{(t)} = \frac{1}{n_j^{(t)}}\sum_{i:z_i^{(t)}=j}x_i$\\
\(\mathcal{O}(nkd)\)  per it., worst-case exponential it. 

Conv. proof: \(\hat{R}(\mu, z) := \sum_{i = 1}^n ||x_i - \mu_{z_i}||_2^2\).
\(\hat{R}(\mu^{(t)}, z^{(t)}) \geq \hat{R}(\mu^{(t)}, z^{(t+1)}) \geq \hat{R}(\mu^{(t+1)}, z^{(t+1)})\) 
\subsection*{k-mean++}
- Start with random data point as center\\
- for $j=2$ to $k$:
$i_j$ sampled with prob.\\
$P(i_j=i) = \frac{1}{z} \underset{1\leq l<j}{min}||x_i-\mu_l||_2^2$; $\mu_j \leftarrow x_{i_j}$

in exp. conv. to \(\mathcal{O}(\log k) \cdot \)OPT

Selecting \(k\): elbow method, regularization

\input{chapters/08_DimensionReduction}
\section*{Probability Modeling}
Assumption: Data set is generated iid\\
Find $h:X\rightarrow Y$ that minimizes pred. error 

$\hat{y} =h^*(x) = \mathbb{E}[Y|X=x]$ for sq. loss

\subsection*{Maximum Likelihood Estimation (MLE)}
Choose a particular parametric $\hat{p}(Y|X,\theta)$

\(\theta^* = \underset{\theta}{\operatorname{amax}} \hat{p}(y_{1:n}|x_{1:n},\theta)\\\iid 
    \operatorname{amin_\theta} - \sum_{i=1}^n log \hat{p}(y_i|x_i,\theta)\)

\subsection*{Ex. Conditional Linear Gaussian}

Assume Gaussian noise $y = f(x) + \epsilon$ with $\epsilon \sim \mathcal{N}(0, \sigma^2)$ and $f(x) = w^\top x$:

\qquad \qquad $\hat p(y \; | \; x, \theta) = \mathcal{N}(y; w^\top x, \sigma^2)$

The optimal $\hat w$ can be found using MLE:

$\hat w = \argmax{w} \; p(y | x, \theta) =\argmin{w} \sum (y_i - w^\top x_i)^2$
\subsection*{Maximum a Posteriori Estimate}

Assume $y = f(x; \theta^*) + \epsilon$, $\epsilon \sim \mathcal{N}(0, \sigma^2)$, but $\theta^* \sim \mathcal{N}(0, \sigma_\theta^2 I_d)$ The posterior distribution of $\theta$ is given by:
$p(\theta \; | \; \mathcal{D}) = \frac{p( \mathcal{D} \; | \; \theta)}{p( \mathcal{D})} \cdot p(\theta)$

Now we want to find the MAP for $\theta$:

$\hat \theta = \text{argmax}_\theta \; p(\theta \; | \; \mathcal{D})$

\quad $= \text{argmax}_\theta \; p(\mathcal{D} \; | \; \theta) \cdot p(\theta)$

\quad $= \text{argmin}_\theta - \sum_{i=1}^n \log p(y_i \; | \; x_i, \theta) - \log p(\theta)$

\quad $= \text{argmin}_\theta \; \frac{\sigma^2}{\sigma_\theta^2} ||\theta||_2^2 + \sum_{i=1}^n(y_i - f(x_i; \theta))^2$

Regularization can be understood as MAP inference, with different priors (= regularizers) and likelihoods (= loss functions).


\subsection*{Statistical Models for Classification}

$f$ minimizing the population risk: $f^*(x) = \text{argmax}_{\hat y} \; p(\hat y \; | \; x)$

This is called the Bayes' optimal predictor for the 0-1 loss. Assuming iid. Bernoulli noise, the conditional probability is:

\qquad \qquad$p(y \; | \; x,w) \sim \text{Ber}(y; \sigma(w^\top x))$

Where $\sigma(z) = \frac{1}{1 + \exp(-z)}$ is the sigmoid function. Using MLE we get:

\quad \;$\hat w = \argmin{w} \sum_{i = 1}^n \log (1 + \exp(-y_i w^\top x_i))$

Which is the logistic loss. Instead of MLE we can estimate MAP, e.g. with a Gaussian prior:

$\; \;\hat w = \argmin{w} \; \lambda ||w||_2^2 + \sum_{i = 1}^n \log (1 + e^{-y_i w^\top x_i})$

% \subsection*{Logistic regression}
% Assume iid Bernoulli noise instead of Gauss.\\
% $P(y|x,w) = Ber(y; \sigma(w^Tx)) = \frac{1}{1+exp(-y w^T x)}$\\
% $l_{logistic}(w;x_i,y_i)=log(1+exp(-y_iw^Tx_i))$\\
% $\nabla_wl(w)=\frac{(-y_ix_i)}{1+exp(+y_iw^Tx_i)}=P(Y=-y|x,w)(-y_ix_i)$
% %$=\begin{cases}
% %1/(1+exp(-w^Tx)) = \sigma(w^T x)\\
% %		1 - 1/(1+exp(-w^Tx)) = \sigma (-w^T x)\\
% %\end{cases}$
% %Learning: $w = \underset{w}{\operatorname{argmax}} P(w|x,y)$\\
% %Classification: Use $P(y|x,w) = \frac{1}{1+exp(-yw^Tx)}$ and predict most likely class label.

% \subsection*{Example: MLE for logistic regression}
% $\operatorname{argmax_w} P(y_{1:n}|w,x_{1:n})\\
% = \operatorname{argmin_w} - \sum_{i=1}^n log P(y_i|w,x_i)\\
% = \operatorname{argmin_w} \sum_{i=1}^n log(1+exp(-y_i w^T x_i))\\
% \hat{R}(w) = \sum_{i=1}^n log(1+exp(-y_i w^T x_i))$ (neg log l. f.)
% %negative log likelihood function

% %\subsection*{Gradient for logistic regression}
% %Loss function $l(w) = log(1+exp(-yw^Tx))$\\
% %$\nabla_w l(w) = \frac{1}{1+exp(-yw^Tx)} exp(-yw^Tx) (-yx)$\\
% %$=\frac{1}{1+exp(yw^Tx)} (-yx)$\\
% %$=P(Y = -y|w, x) (-yx)$

% \subsection*{Logistic regression and regularization}
% $\underset{w}{\operatorname{min}} \sum_{i=1}^n log(1+exp(-y_i w^T x_i)) + 
% \lambda (||w||_1 or ||w||_2^2)$

% \subsection*{SGD for logistic regression}
% \iffalse
% 1. Initialize w\\
% 2. For t=1,2,...\\
% Pick data $(x,y) \in_{u.a.r} D$\\
% Prob. of misclas. $\hat{P}(Y = -y|w,x) = \frac{1}{1+exp(yw^Tx)}$\\
% \fi
% Update $w \leftarrow w + \eta_t y x \hat{P}(Y = -y|w,x)$\\
% \textbf{L2 regularized logistic regression:}\\
% Update $w \leftarrow w (1-2\lambda \eta_t) + \eta_t y x \hat{P}(Y = -y|w,x)$

% \subsection*{Multiclass Logistic Regression}
% $P(Y=i|x,w_1,..,w_c)=exp(w_i^Tx)/\Sigma_j^c exp(w_j^Tx)$

\input{chapters/10_DecisionTheory}
\input{chapters/11_Generative}
\input{chapters/12_GMM}
\section*{Large Language Models}
\textbf{Sequence-to-sequence:} Use RNN with hidden state, keep hidden state, encoder: 
current input + last hidden state $\rightarrow$ new hidden state, decoder: 
current hidden state $\rightarrow$ output token + new hidden state

\subsection*{Transformers}
Use encoder/decoder architecture, don't need recurrence because of (self-) attention (multi-head), 
encoder: self-attention + feed-forward (FCNN), decoder: self-attention, encoder-decoder attention, 
feed-forward

\textbf{Self-attention:} For the tokens $X$, generate 
\textbf{query} $Q = X \times W_Q$, \textbf{key} $K = X \times W_K$, \textbf{value} $V = X \times W_V$. 
For each token, perform dot product of query and key, use soft-maxed version to scale value, 
i.e. $Z = \text{softmax}(\frac{Q \times K\top}{\sqrt{d_k}}) V$

\textbf{Positional Encodings:} Add to word embeddings, e.g. sine functions w/ different freq.

\textbf{Enc.-Dec. Att.:} dec.: $Q$ so far. enc.: $K, V$

\textbf{Task}
First, plug in the complete data log-likelihood into the equation for $Q$:

\(Q(\lambda ; \lambda^{j}) := \mathbb{E}_{z_{1:m}}[(n+m)\log(\lambda) - \lambda \sum_{i=1}^mt_i - \lambda \sum_{i=1}^mz_i \ \ | t_{1:n} ; \lambda^{(j)}]\)

The 1. and 2. summand don't depend on $z_{1:m}$ nor $t_{1:n}$ nor $\lambda^{(j)}$:

\(= (n+m)\log(\lambda)+ \lambda \sum_{i=1}^m t_i+\mathbb{E}_{z_{1:m}}[\lambda\sum_{i=1}^mz_i | t_{1:n} ; \lambda^{(j)}]\)



The third summand does not depend on $t_{1:n}$ but\(z_i \sim \)Exp($\lambda^{(j)}$). 
\(\mathbb{E}_{z_{1:m}}[\lambda\sum_{i=1}^mz_i | t_{1:n} ; \lambda^{(j)}] = \lambda\sum_{i=1}^m \mathbb{E}_{z_{1:m}}[z_i | \lambda^{(j)}]\)

We know that $z_i \geq \tau$. We use a new Exp($\lambda^{(j)}$) distributed variable $z_i'$ to model this:

\(\mathbb{E}_{z_{1:m}}[z_i | \lambda^{(j)}] = \mathbb{E}[z_i' | z_i' \geq \tau ; \lambda^{(j)}]\)

\(= \frac{1}{\mathbb{P}[z_i' \geq \tau]} \int_{\tau}^{\inf} z_i' \lambda^{(j)} e^{-\lambda^{(j)}z_i'} dz_i'\) (below $\tau$ the value is $0$)

\(= \frac{1}{\mathbb{P}[z_i' \geq \tau]} (\tau\lambda^{(j)}+1)e^{-\tau\lambda^{(j)}}\frac{1}{\lambda^{(j)}}\) (hint)

Furthermore:

\(\mathbb{P}[z_i' \geq \tau]= \int_{\tau}^{\inf} \lambda^{(j)} e^{-\lambda^{(j)}z_i'}dz_i' = e^{-\lambda^{(j)}\tau}\)

Thus we get:

\(\frac{1}{e^{-\lambda^{(j)}\tau}} (\tau\lambda^{(j)}+1)e^{-\tau\lambda^{(j)}}\frac{1}{\lambda^{(j)}}= (\tau\lambda^{(j)}+1)\frac{1}{\lambda^{(j)}}\)

Assembling this back together, we get for the third summand:

\(\lambda\sum_{i=1}^m \mathbb{E}_{z_{1:m}}[z_i | \lambda^{(j)}] = \lambda^{(j)}\sum_{i=1}^m ((\tau\lambda^{(j)}+1)\frac{1}{\lambda^{(j)}}) = \lambda^{(j)} m (\tau\lambda^{(j)}+1)\frac{1}{\lambda^{(j)}} = m (\tau\lambda^{(j)}+1) = m\lambda^{(j)} (\tau + \frac{1}{\lambda^{(j)}})\)

Finally, summing the three summands again:

\(= \mathbb{E}_{z_{1:m}}[(n+m)\log(\lambda) | t_{1:n} ; \lambda^{(j)}] - \mathbb{E}_{z_{1:m}}[\lambda \sum_{i=1}^mt_i | t_{1:n} ; \lambda^{(j)}] - \mathbb{E}_{z_{1:m}}[\lambda\sum_{i=1}^mz_i | t_{1:n} ; \lambda^{(j)}]\)

\(= (n+m)\log(\lambda) - \lambda \sum_{i=1}^m t_i - m\lambda^{(j)} (\tau + \frac{1}{\lambda^{(j)}})\)


\end{multicols*}
\end{document}