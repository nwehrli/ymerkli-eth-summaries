\documentclass[11pt,landscape,a4paper,fleqn]{article}
\usepackage[utf8]{inputenc}
\usepackage[ngerman]{babel}
\usepackage{tikz}
\usetikzlibrary{shapes,positioning,arrows,fit,calc,graphs,graphs.standard}
\usepackage[nosf]{kpfonts}
\usepackage[t1]{sourcesanspro}
%\usepackage[lf]{MyriadPro}
%\usepackage[lf,minionint]{MinionPro}
\usepackage{multicol}
\usepackage{wrapfig}
\usepackage[top=0mm,bottom=3mm,left=1mm,right=1mm]{geometry}
\usepackage[framemethod=tikz]{mdframed}
\usepackage{microtype}
\usepackage{mathtools}
\usepackage{ccicons}
\usepackage{hyperref}

\let\bar\overline
\def\D{\mathcal{D}}

\definecolor{myblue}{cmyk}{1,.72,0,.38}
\definecolor{myorange}{cmyk}{0,0.5,1,0}

\pgfdeclarelayer{background}
\pgfsetlayers{background,main}

\everymath\expandafter{\the\everymath \color{myblue}}
\everydisplay\expandafter{\the\everydisplay \color{myblue}}

\renewcommand{\baselinestretch}{.8}
\pagestyle{empty}

\global\mdfdefinestyle{header}{%
linecolor=gray,linewidth=1pt,%
leftmargin=0mm,rightmargin=0mm,skipbelow=0mm,skipabove=0mm,
}

\newcommand{\header}{
\begin{mdframed}[style=header]
\footnotesize
\sffamily
Cheat sheet\\
Nicolas Wehrli,~page~\thepage~of~2
\end{mdframed}
}

\makeatletter
\renewcommand{\section}{\@startsection{section}{1}{0mm}%
                                {0pt}%
                                {0.5pt}%x
                                {\color{myorange}\sffamily\small\bfseries}}
\renewcommand{\subsection}{\@startsection{subsection}{1}{0mm}%
                                {0pt}%
                                {0.1pt}%x
                                {\sffamily\bfseries}}


\makeatother
\setlength{\parindent}{0pt}

\newcommand{\red}[1]{\textcolor{red}{#1}}
\newcommand\iid{\stackrel{\mathclap{\normalfont\mbox{\tiny{iid}}}}{=}}

\begin{document}
	
\section*{Disclaimer}

\small
\begin{multicols*}{4}
	
This document is an exam summary that follows the slides of the \textit{Introduction to Machine Learning} lecture  at ETH Zurich. 
The contribution to this is a short summary that includes the most important concepts, formulas and algorithms. 
This summary was created during the spring semester 2018 by Yannik Merkli and adapted in 2024 by Nicolas Wehrli. 
Due to updates to the syllabus content, some material may no longer be relevant for future versions of the 
lecture. This work is published as CC BY-NC-SA.

\begin{center}
	\ccbyncsa
\end{center}

I do not guarantee correctness or completeness, nor is this document endorsed by the lecturers. 
Feel free to point out any erratas. 
For the full \LaTeX \ source code, consider \texttt{\href{https://github.com/nwehrli/ymerkli-eth-summaries}{https://github.com/nwehrli/ymerkli-eth-summaries}}.

\newpage
	
%\section*{Probabilities}
$P(X,Y)=P(X|Y)P(Y)=P(Y|X)P(X)\\
P(X,Y|Z)=P(X|Y,Z)P(Y|Z)\\
P(X,Y|Z)=P(Y|X,Z)P(X|Z)$\\
$\text{X,Y iid:}P(X,Y|Z)=P(X|Z)P(Y|Z)$\\
$P(X=x)=\sum_{y'\in Y}P(X=x,Y=y')$\\
$\text{iid:}P(X_1,...,X_n|Y)=\prod_{i=1}^{n}P(X_i|Y)$\\
$\mathbb{E}_x[X] = \begin{cases}
   \int x \cdot p(x) \partial x  &|\mathbb{E}_x[f(x)] =\\
   \sum_x x \cdot p(x) &|\int f(x) \cdot p(x) \partial x
  \end{cases}$\\
%$\mathbb{E}_x[f(x)] = \int f(x) \cdot p(x) \partial x $\\
$\mathbb{E}[X+Y]=\mathbb{E}[X]+\mathbb{E}[Y]$\\
$\sigma_X^2=Var[X] = \mathbb{E}[(X-\mu_X)^2] = \mathbb{E}[X^2] - \mathbb{E}[X]^2$\\
$p(Z|X,\theta) = \frac{p(X,Z|\theta)}{p(X|\theta)}$\\
%\subsection*{Linearity of expectation}
%$X, Y$ rand. var., $a, b \in \mathbb{R}$:\\
%$\mathbb{E}_{x,y}[aX + bY] = a\mathbb{E}_x[X] + b \mathbb{E}_y[Y]$
\section*{Basics}
%-------------------------------------------------------------------------------------------
%PROBABILITIES
%-------------------------------------------------------------------------------------------
% $P(X,Y)=P(X|Y)P(Y)=P(Y|X)P(X)\\
% P(X,Y|Z)=P(X|Y,Z)P(Y|Z)\\
% P(X,Y|Z)=P(Y|X,Z)P(X|Z)$\\
% $P(X,Y,Z)=P(X|Y,Z)P(Y|Z)P(Z)$\\
% $\text{X,Y iid:}P(X,Y|Z)=P(X|Z)P(Y|Z)$\\
% $P(X=x)=\sum_{y'\in Y}P(X=x,Y=y')$\\
% $\text{iid:}P(X_1,...,X_n|Y)=\prod_{i=1}^{n}P(X_i|Y)$\\
% $\mathbb{E}_x[X] = \begin{cases}
% \int x \cdot p(x) \partial x  &|\mathbb{E}_x[f(x)] =\\
% \sum_x x \cdot p(x) &|\int f(x) \cdot p(x) \partial x
% \end{cases}$\\
% %$\mathbb{E}_x[f(x)] = \int f(x) \cdot p(x) \partial x $\\
% $\mathbb{E}[X+Y]=\mathbb{E}[X]+\mathbb{E}[Y]$\\
% $\sigma_X^2=Var[X] = \mathbb{E}[(X-\mu_X)^2] = \mathbb{E}[X^2] - \mathbb{E}[X]^2$\\
% $p(Z|X,\theta) = \frac{p(X,Z|\theta)}{p(X|\theta)}$
%\subsection*{Linearity of expectation}
%$X, Y$ rand. var., $a, b \in \mathbb{R}$:\\
%$\mathbb{E}_{x,y}[aX + bY] = a\mathbb{E}_x[X] + b \mathbb{E}_y[Y]$

%-------------------------------------------------------------------------------------------
%Calculus and stuff
%-------------------------------------------------------------------------------------------


% $ln(x) \leq x - 1, x>0$; $||x||_2 = \sqrt{x^T x}$; $\nabla_x ||x||_2^2 = 2 x$%; $||x||_p = (\sum_{i=1}^n|x_i|^p)^{\frac{1}{p}}$, $1 \leq p < \infty$

% $f(x) = x^T A x$; $\nabla_x f(x) = (A + A^T) x$

% $D_{KL} = \mathbb{E}_p[log(\frac{p(x)}{q(x)})]$; $D_{KL} (P||Q) = \sum_{x \in X}P(x) \cdot log \frac{P(x)}{Q(x)} =  \int_{-\infty}^{+\infty} p(x) log \frac{p(x)}{q(x)} \, dx $ always nonneg

\textbf{Orth:} A: $det(A)\in\{+1,-1\},AA^T=A^TA=I$\\
\(trace(ABC) = trace(BCA) = trace(CAB)\)\\ 
\(trace(A)=\sum \lambda_i(A);
% $, A\in\mathbb{R}^{n\times n}, (A^{-1})^T=(A^T)^{-1}$\\
% $rank(A)=n, det(A)\neq0$\\
\begin{bmatrix}
a&b \\ 
c&d
\end{bmatrix}^{-1}=\frac{1}{ad-bc}
\begin{bmatrix}
d&-b \\ 
-c&a
\end{bmatrix}
\)

\(A=\sum_{k=1}^{rk(A)}\sigma_{k,k}u_k (v_k)^T, A^\dag = U S' V^T; \sigma'_{k, k} = \frac{1}{\sigma_{k,k}}\)

\textbf{Deriv:}
$\frac{\partial}{\partial x}b^Tx=\frac{\partial}{\partial x}x^Tb=b^T,\!
\frac{\partial}{\partial x}||x||_2^2=2x^T,\! \frac{\partial}{\partial x} ||x -a||_2 = \frac{(x-a)^T}{||x-a||_2},\!
\frac{\partial}{\partial x}(x^TAx)=x^T(A^T+A),$
$\frac{\partial}{\partial x}(b^TAx)=A^Tb, \nabla_X(c^TXb)=cb^T,
\nabla_X(c^TX^Tb)=bc^T; ||A||_{op} = sup_{||x||_2=1}||Ax||_2$

$\textbf{convex} \iff f(\lambda x+ (1-\lambda)y) \leq \lambda f(x) + (1-\lambda)f(y); 
f(y) \geq f(x) + \langle\nabla f(x), y-x\rangle; D^2f(x) \geq 0$\\
\(\alpha f + \beta g \textbf{ c.}; \max(f, g) \textbf{ c.} \text{ if } f, g \textbf{ c.}, \alpha, \beta \geq 0\)\\
\(f \circ g = f(g(x)) \textbf{ c.} \text{ if } f \textbf{ c.}, g \textbf{ a.} \lor f \textbf{c., non-dec.}, g \textbf{ c.}\)
% \textbf{Eigdec:}
% $A,Q \in \mathbb{R}^{n\times n}, A=Q\Lambda Q^{-1},\! \Lambda = diag(\lambda_i)$\\
% $Q=[v_1,..,v_n], \text{(col's are e-vec.)}$\\
% $\text{if all $\lambda_i\geq0:$} A^{-1}=Q\Lambda^{-1}Q^{-1},\Lambda^{-1}=diag(\frac{1}{\lambda_i})$\\
% $\text{if }A=A^T\text{(symm.) and }x^TAx\geq0 \forall x \neq 0 \rightarrow psd$\\

% $X\in \mathbb{R}^{n\times p}, U\in \mathbb{R}^{n\times n}, S\in \mathbb{R}^{n\times p},
% V\in \mathbb{R}^{p\times p}$\\
% $X^TX=VS^TU^TUSV^T=VS^TSV^T=V\Sigma V^T$\\
% $\Sigma = diag(\sigma_1^2,..,\sigma_n^2);\sigma_i^2=\lambda_i; \forall \lambda_i \geq 0$
\(p_{\mu, \Sigma}(x)=\frac{1}{\sqrt{(2\pi)^p det(\Sigma)}}\exp(-\frac{1}{2}(x-\mu)^T\Sigma^{-1}(x-\mu))\)\\
\(\mathbf{X} \sim \mathcal{N}(\mu, \Sigma) \implies A\mathbf{X} + b \sim \mathcal{N}(A\mu+b, A\Sigma A^T)\)

% \textbf{Gauss CDF:}\\
% $\Phi(u;v,w) = \int_{-\infty}^{u} \mathcal{N}(y;v,w)dy=\Phi(\frac{u-v}{\sqrt{w}};0,1)$;\\ %CDF: cumulative distribution function; PDF: standard normal probability density function, $\mu = 0$, $\sigma = 1$
%PDF: $\phi(x) = \frac{1}{\sqrt{2\pi}} e^{-(1/2)x^2}$; $\int \phi(x) \partial x = \Phi(x) + c$;\\
%$\int x \phi(x) = -\phi(x) + c$; $\int x^2 \phi(x) \partial x = \Phi(x) -x \phi(x) + c$
\textbf{Jensen ineq: }$g(E[X]) \leq E[g(X)]$, $g$ convex


\section*{Regression}
\subsection*{Linear Regression $f(x)=w^Tx; X \in \mathbb{R}^{n \times d}$}
$L(w) = ||Xw-y||^2_2; X^TX \hat{w} = X^T y$\\
% $\hat{w} = \operatorname{argmin_w} \sum_{i=1}^n (y_i - w^Tx_i)^2$\\
\(d \leq n: \hat{w} = (X^TX)^{-1}X^Ty\) if \(rk(X) = d\)\\
\(n < d: \hat{w} = (X^TX)^\dag X^T y\); \(rk(X) = n\) \(||\hat{w}||_2\) min.

$\nabla_w L(w)  = 2X^T (Xw-y)$
% = -2 \sum_{i=1}^n (y_i-w^T x_i) \cdot x_i
\subsection*{Gradient Descent}
1. Start arbitrary $w_o \in \mathbb{R}$\\
2. Do $w_{t+1} = w_t - \eta \nabla L(w_t)$ until \(||w^t - w^{t-1}||_2 \leq \epsilon\)

GD conv. to \(\hat{w}\) if \(rk(X^TX) = d, \eta < \frac{2}{\lambda_{max}(X^TX)}\)
\(||w^{t+1} - \hat{w}||\leq ||I - \eta X^TX||_{op}||w^t-\hat{w}||_2 \leq 
\rho^{t+1}||w^0-\hat{w}||_2; \eta_{opt} = \frac{2}{\lambda_{max}+\lambda_{min}}; 
\rho_{min} = 1-\eta_{opt}\lambda_{min} = \frac{\kappa-1}{\kappa+1}\)
\textbf{minibatch SGD:} \(\nabla L_S(w)\) on random \(S \subset D\) every iter.; \(|S| = 1\) SGD; 
\(E_S(\nabla L_S(w)) = \nabla L(w)\)

\textbf{strictly c.} \(\implies\) stationary point is unique g. min.;
\textbf{strongly c.} \(\implies\) unique g. min. exists 
\subsection*{Errors}
exp. estim. err.: \(E_X(\ell(f(X), f^*(X)))\); \(y = f^*(x)+\varepsilon\)\\
generaliz. err.: \(L(f; \mathbb{P}_{X,Y}) = E_{X,Y}(\ell(f(X), Y))\)
\(L(\hat{f}; \mathbb{P}_{X,Y}) = E_X((\hat{f}(X) - f^*(X))^2) + \sigma^2\)(sq. loss)
\(L(\hat{f}_\D; \D_{\text{test}}) = 
\frac{1}{|\D_{\text{test}}|}\sum_{(x,y) \in \D_{\text{test}}}
\ell(\hat{f}_\D(x),y)\) estim. generaliz. err.; g. err. + const = exp. estim err.
\textbf{k-fold CV:} \(\uparrow\)k \(\implies\) \(\hat{f}_{M_i, \D'} \approx \hat{f}_{M_i, \mathcal{D_{\text{use}}}}\), 
\(CV_k(M_i) \not \approx L(\hat{f}_{M_i, \D_\text{use}};\mathbb{P}_{X, Y})\); extreme: LOOCV
\subsection*{Bias-Variance Tradeoff}
\(\text{Bias}_{\D}^2(\hat{f}_\D, x) := (E_\D(\hat{f}_\D(x))-f^*(x))^2\)\\
\(\text{Bias}^2_{\D}(\hat{f}_\D) := E_X(\text{Bias}_{\D}^2(\hat{f}_\D, X));
\text{Var}_\D(\hat{f}_\D) := E_X(\text{Var}_\D(\hat{f}_\D(X))); 
E_\D(L(\hat{f}_\D; \mathbb{P}_{X,Y})) = \text{Var}_\D(\hat{f}_\D)+\text{Bias}_\D^2(\hat{f}_\D)+\sigma^2\)
% \subsection*{Gaussian/Normal Distribution}
% $\sigma =$ standard deviation, $\sigma^2 =$ var., $\mu =$ mean:\\
% $f(x) = \frac{1}{\sqrt{2\pi\sigma^2}} exp(-\frac{(x-\mu)^2}{2\sigma^2})$\\
% $f(x_1,.,x_k)=\frac{1}{\sqrt{(2\pi)^k|\Sigma|}}
% exp(-\frac{1}{2}(\boldsymbol{x-\mu})^T\Sigma^{-1}(\boldsymbol{x-\mu}))$

\textbf{Ridge} closed form: $\hat{w}=(X^T X + \lambda I)^{-1} X^T y$

%\subsection*{L1-regularized regression (the Lasso)}
%Regularization: $\underset{w}{\operatorname{min}} \sum \limits_{i=1}^n (y_i - w^Tx_i)^2 + \lambda ||w||_1$\\
%Encourages coefficients to be exactly 0.

% \subsection*{Standardization}
% Goal: each feature: $\mu = 0$, unit $\sigma^2$: $\tilde{x}_{i,j} = \frac{(x_{i,j}-\hat{\mu}_j)}{\hat{\sigma}_j}$\\
% $\hat{\mu}_j = \frac{1}{n}\sum_{i=1}^n x_{i,j}$, $\hat{\sigma}_j^2 = \frac{1}{n}\sum_{i=1}^n {(x_{i,j}-\hat{\mu}_j)}^2$ 




%\subsection*{Regularization}
%The error term $L$ and the regularization $C$ with regularization parameter $\lambda$: $\min \limits_w L(w) + \lambda C(w)$\\
%L1-regularization for number of features \\
%L2-regularization for the length of $w$

%my idea
%\subsection*{Regularization}
%A lot of supervised learning problems can be written in this way: $\lambda$: $\min \limits_w L(w) + \lambda C(w)$\\
\input{Classification}
\input{Kernels}
\input{Imbalance}
\input{Multi-class}
\input{NeuralNetworks}
\input{Clustering}
\input{DimensionReduction}
\input{ProbabilityModeling}
\input{DecisionTheory}
\input{Generative}
\input{Latent}

\end{multicols*}
\end{document}